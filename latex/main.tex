\documentclass{article}
% Margin definition.
\usepackage[a4paper,total={6.8in, 8.5in}]{geometry}
\usepackage{parskip}
% Images.
\usepackage{graphicx}
\usepackage[hidelinks, bookmarks=true]{hyperref}
\usepackage{float}
% Encoding.
\usepackage[english]{babel}
\usepackage[utf8]{inputenc}
% Allow multiline comments
\usepackage{verbatim} 
% Helvetic font.
\usepackage[scaled]{helvet}
\renewcommand\familydefault{\sfdefault} 
% Header for UA logo.
\usepackage{fancyhdr}
% Dots in index.
\usepackage[titles]{tocloft}
\renewcommand{\cftsubsecleader}{\Large\cftdotfill{0}}
\renewcommand{\cftsecleader}{\Large\cftdotfill{0}}
\renewcommand{\cftsecfont}{\large\bfseries\scshape}
\renewcommand{\cftsubsecfont}{\scshape}
\renewcommand*{\HyperDestNameFilter}[1]{\jobname-#1}
% Dot after number in (sub)sections and in toc.
\renewcommand{\cftsecaftersnum}{.}
\renewcommand{\cftsubsecaftersnum}{.}
\usepackage{titlesec}
\titlelabel{\hspace{-0.5cm}\quad}
\usepackage[letterspace=45]{microtype}
\newcommand*{\fullref}[1]{\hyperref[{#1}]{\autoref*{#1} \nameref*{#1}}}
% Header with UA logo definition. 
\pagestyle{fancy}
\fancyhf{}
\chead{
    \includegraphics[width=5in]{./images/header_ua.png}
}
\setlength\headheight{20pt}
% Footer with page number.
\rfoot{Page \thepage}
\renewcommand{\footrulewidth}{0.1pt}
% Rename table of contents title to "Index"
\renewcommand{\contentsname}{\normalsize Index \vspace{0.6cm}}
% Add text with hyperlink
\usepackage{hyperref}
%\hypersetup{
%    colorlinks=true,
%    linkcolor=blue,
%    filecolor=magenta
%}
% Water mark
\newsavebox\mybox
\usepackage[printwatermark]{xwatermark}
\usepackage{xcolor}
\usepackage{tikz}
% paragraph
\newcommand\tab[1][1cm]{\hspace*{#1}}
\setlength\parindent{24pt}
%images
 \usepackage{graphicx}
\usepackage{caption}
% footnotes at bottom
\usepackage[bottom]{footmisc}
% Urls with line break
\usepackage{pdflscape}
% Drawing functions
\usepackage{tikz}
\usepackage{pgfplots}
\pgfplotsset{width=7cm, height=4cm, compat=1.17}

\usepackage{multicol}
\setlength{\columnsep}{1cm}

%%%%%%%%%%%% References/Bibliography %%%%%%%%%%%%
\usepackage{biblatex}
\addbibresource{bibliography.bib}

%%%%%%%%%%%%%%%%%%%%%%%%%%%%%%%%%%%%%%%%%%%%%%%%%

\begin{document}

%%%%%%%%%%%%%%%%%% Cover Page %%%%%%%%%%%%%%%%%%
\title{\vspace{-0.9cm}
       \vspace{1cm}
       \normalsize
       \raggedright\textbf{Title: \hspace{1.5cm} Bitcoin Mining 101} \\ \vspace{0.4cm}
       \raggedright\textbf{Author: \hspace{1.1cm} Eduardo Santos} \\ \vspace{0.4cm}
       \raggedright\textbf{Date: \hspace{1.47cm} 11/06/2021} \\}
\author{}
\date{}

\maketitle
\thispagestyle{fancy}

%%%%%%%%%%%%%%%%%% END Cover Page %%%%%%%%%%%%%%%%%%

\vspace{-1.4cm}

\tableofcontents


\fontsize{10pt}{13pt}
\selectfont
\lsstyle

\titlelabel{\thetitle.\quad}	

\newpage

\section{Introductory Note}

\tab This assignment focus on Bitcoin mining, and will answer the following questions:

\begin{itemize}
    \item What is Bitcoin mining?
    \item How does Bitcoin Mining work?
    \item Is mining Bitcoins worth it?
    \item What do I need to mine bitcoins?
\end{itemize}
\tab 

\section{Summary / Abstract}

\tab This assignment's objective is not to explain what Bitcoin or other cryptocurrencies are, as it assumes that we already know that. 

Instead, it is related to  Bitcoin mining, trying to explain them in the simplest possible way, so that anyone can understand how it works and, if wanted, can start to mine Bitcoins right away.

\section{Framework}

\tab Since the explosion of Bitcoin and other cryptocurrencies, there is one question that has undoubtedly been on most of investor's mind, and that question is: "How can I mine Bitcoin and is it worth it?"

This cryptocurrency's evolution is something that we cannot ignore, since it has been extermely spoked about since April of 2011, time on which its value exploded, with a gain of 3200\% within three short months.

This question is the main premise for this report.

\newpage

\section{Bitcoin Mining}
\subsection{What is Bitcoin Mining?}

\tab Bitcoin mining is the process of updating the ledger of Bitcoin transactions known as the Blockchain. This process is usually performed by computers with a lot of processing power, such that they are able to calculate in a way that we humans could not, at least by hand. 

The expression "Bitcoin mining" derives from an analogy relating to how people physically mine materials in the real world. We can think of it like this, mining is the process of extracting finite resources from our planet, this is much like Bitcoin mining, which put simply is the process of extracting finite Bitcoins through the Bitcoin network. So how does that work?

\subsection{How does Bitcoin Mining work?}

\tab By definition, Bitcoin mining is the process of generating new Bitcoin by solving algorithmic puzzles. The people who solve these puzzles on the Bitcoin network are referred to as Miners.

These miners highly advanced computers that are able to run complex mathematical algorithms, ultimately this algorithms are really just a long series of numeric values that, when ordered correctly, result in the puzzle being solved.

When a miner solves a puzzle, they are rewarded in Bitcoin. The amount of Bitcoin a miner receiver upon completion of a puzzle is referred to as the \textit{Bitcoin reward per block}, because, whenever a puzzle is solved, it adds a new block onto the Bitcoin blockchain, which in turn affects the whole network.

\subsection{The impact of mining in the Blockchain}

\tab The Blockchain refers to the decentralized ledger that records all Bitcoin related transactions worldwide, whenever a transaction is made, or new Bitcoin is added to the network, the ledger is updated and a new block is added to the chain. This means that each block related to an update in the ledger, either due to a transaction, or the addition of new Bitcoin.

Therefore, when a puzzle is solved and a Bitcoin has been generated, the ledger is updated, this results in a new block being added to the chain, hence the name \textit{reward per block}. Not only does mining allows the miner to produce and earn Bitcoins, it is also an essential activity that allows the ledger of transactions upon which Bitcoin is based to be maintained.

Now that we understand how people mine and how does this influences the Blockchain, how much are miners rewarded when they solve a puzzle?

The reward per block was 50 Bitcoin, back in 2008, however, the rewards is halved every 210 000 blocks that are solved. Blocks are solved at an average of one block every 10 minutes, this means that the reward per block halves around every 4 years. The most recent halving occurred in July of 2020, meaning that current reward per block is 6.25 Bitcoin.

All recent reports state that there is roughly 18 million Bitcoin currently in circulation, However, we can cross-reference this figure by following this calculation:

\[(50 x 210 000) + (25 x 210 000) + (12.5 x 210 000) + (6.25 x 210 000) = 18 375 000\]

 By calculating this number, we are able to confirm that there are at least \(18 375 000\) Bitcoin in circulation as of July 2020.
 
 So how does the formula of halving the Bitcoin reward after each 210 000 puzzles solved, effect the long-term generation of Bitcoin?
 When Bitcoin was created, there were many stipulations enforced by specific coding, one of this rules is that the total number of Bitcoin must be limited and therefore have a finite suply. The cap was set so that, when the final puzzle is solved, there would only ever be a cumulative amount of 21 million Bitcoins. By following the same logic that we applied how much Bitcoin is currently in circulation, we can also estimate when the last Bitcoin will be mined.
 
 The main factor to consider is that the reward is halved once in every 4 year increment, so the final mining date can be determined as follows: In 2008, the initial reward was 50 Bitcoin per block. So, to find out when Bitcoin will be depleted, we just need to keep on halving 50 until we reach 0, meaning that there is no reward per block because there is no Bitcoin left to be mined. 50 can be halved 33 times before it reaches 0, and there have been 3 halvings already. This means that there are 30 halvings remaining before Bitcoin has been depleted.
 
 As a halving occurs every 4 years, we can multiply 30 by 4 to find out how long it will be until the final Bitcoin will be mined. The answer to this is 120 years, meaning that the final Bitcoin will be mined inn 2140.
 
 But, if there is already 18 million out of 21 million Bitcoin in circulation, within 12 years of the first Bitcoin being mined, why will it take 120 years to mine the last 3 million?
 
 This is due to the reward per block halving 4 years, the rate at which new Bitcoin is generated drastically decreases. When the final division takes effect, miners will only be rewarded 0.00000001 Bitcoin per block, this means that when the last puzzle is solved, and the final Bitcoin is mined in 2140, no more Bitcoin can ever be generated.
 
 Linking back to the metaphor we used before, this is equivalent to all the mines on this planet being drained and, consequently, the planet Earth being depleted of the limited finite resources that it once had.
 
 So, if this is the case, why isn't everyone mining now, in order to get Bitcoin while reward per block is still considerably high?
 
 

 
 \subsection{Is Bitcoin mining worth it?}
 
 
 
 
 
 





% Add "References" to table of contents
\addcontentsline{toc}{section}{References}
% No cite makes all references appear, even if there's no citation on the text
\nocite{*}
\printbibliography

\end{document}